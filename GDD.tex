\documentclass[10pt, a4paper]{article}

\usepackage[
top=3 cm,
bottom=3 cm,
left=2.5 cm,
right=2.5 cm,
]{geometry}

\usepackage{blindtext}
\usepackage{titlesec}
\usepackage{iftex}
\usepackage{graphicx}
\usepackage{hyperref}

\hypersetup{
	colorlinks=true,
	linkcolor=black,
	filecolor=magenta,      
	urlcolor=blue,
	pdftitle={Game Design Document},
	pdfpagemode=FullScreen,
}

\urlstyle{same}


\ifPDFTeX
\input{./glyphtounicode}
\pdfgentounicode=1
% \usepackage[T1]{fontenc} 
\usepackage[utf8]{inputenc}
\usepackage[sfdefault,light]{inter}
\fi


\title{Game Design Document}
\author{}
\date{}

\begin{document}
	\maketitle
	\tableofcontents
	\pagebreak
	
	\section{Game Overview}
	\subsection{Title} TBD
	\subsection{Genre} Card Battler + Deck Builder
	\subsection{Description} 
	A roguelike card-battler game where players explore a dark, mysterious world filled with arcane tech and ancient magic. Players start with a randomized deck and grow stronger by strategically modifying their deck through battles, exploration, and special nodes (shrines, forges, etc.).
	\subsection{Platforms} Desktop (Windows, Linux, MacOS)
	\subsection{Target Audience}  Fans of Inscryption, Slay the Spire, Griftlands, Monster Train, and dark fantasy universes. Those who enjoy strategic, turn-based gameplay, deckbuilding, and lore-driven worlds.
	\subsection{Art Style} TBD
	\subsection{Audio Style} TBD
	\subsection{Theme} Dark Fantasy + Arcane Sci-Fi
	\subsection{Controls} Mouse + Keyboard
	\pagebreak
	\section{Core Gameplay}
	\begin{itemize}
		\item \subsection{Turn-Based Combat} Players use a hand of cards drawn from their deck to battle enemies in strategic, turn-based encounters. Energy is spent to play cards, and unspent energy may have special synergy effects. Active cards may be sacrificed to gain additional energy.
		
		\item \subsection{Card Stats}
		\begin{itemize}
			\item \textbf{Attack:} Deal direct damage.
			\item \textbf{Health:} Health of the Cards.
			\item \textbf{Energy:} Determines how much energy is required to play the card.
			\item \textbf{Sigil:} Abilities which can either have positive or negative effects.
		\end{itemize}
		
		\item \subsection{Deck Management} 
		\begin{itemize}
		\item Players will have an inventory with acquired cards and a deck. Players can freely exchange cards between the Deck and Inventory outside of battle.
		\item Players can:
		\begin{itemize}
			\item Upgrade cards at special points
			\item Fuse or corrupt cards for alternate versions at certain points.
		\end{itemize}
	    \end{itemize}
	    
		\item \subsection{Map Exploration}
		Each run will have a procedurally generated map.
		\begin{itemize}
			\item \textbf{Combat Encounters:} Normal, Mini Boss, Boss
			\item \textbf{Event Nodes:} Narrative choices with rewards/penalties
			\item \textbf{Shrines:} Buff or debuff cards
			\item \textbf{Forges:} Upgrade, craft, or combine cards
		\end{itemize}
		
		\item \subsection{Characters (Future)} Multiple unlockable characters with distinct starting decks, abilities, and themes.
		
		\item \subsection{Synergies} Cards can synergize with eachother through .
		\end{itemize}
	\section{Meta Progression}
	
	\begin{itemize}
		\item \subsection{Unlockable Cards} New cards added to the global pool after certain milestones or after obtaining them in a run.
		
		\item \subsection{Characters (Future) } Unlock through progression, each with unique starting decks and playstyles.
		
		\item \subsection{Artifacts} Permanent or run-specific passive bonuses (e.g., +1 max energy, double draw on first turn).
		
		\item \subsection{Codex / Lore Archive} Unlocked via choices/events, slowly revealing the overarching narrative.
		
		\item \subsection{Difficulty Scaling (Future)} Ascension-like system that unlocks tougher modifiers as players complete runs.
	\end{itemize}
	\section{Monetization}
	
	\begin{itemize}
		\item \subsection{Open Source Premium Model (PC/Console):} One-time purchase for pre-compiled binaries, no in-game ads or microtransactions. Free access to source code (Aseprite's Monetization Model)
		
		\item \subsection{Planned DLC (Future)}
		\begin{itemize}
			\item New characters, card sets, environments
			\item Alternate campaigns or boss variants
		\end{itemize}
	\end{itemize}
	
	\section{Tech Stack}
		\begin{itemize}
			\item \textbf{Unity 6000.0.48f1} [LTS] - Game Engine
			\subitem \textbf{LeanTween} - Tweening
			\subitem \textbf{Newtonsoft.JSON} - JSON Library
			\item \textbf{WASDEditor} - Dialogues
		\end{itemize}
	\section{Inspirations}
	\begin{itemize}
		\item \href{https://www.inscryption.com}{Inscryption}
		\item \href{https://www.nethack.org}{NetHack}
	\end{itemize}
	
	\section{Links}
	\begin{itemize}
		\item \href{https://www.figma.com/design/Zb3b7Vf4XvZ4UPItO0jaNE/Deck-Builder}{Design Moodboard}
		\item \href{https://docs.google.com/spreadsheets/d/16LLjNDr2vdsV7ZZCh_UFrKlwE__lp5mAs24b4kD_r_s/edit?usp=sharing}{Brainstorming \& Balancing Sheet}
	\end{itemize}
	\section{Feasibility}
	A barebones "First Draft" is possible to do with a couple hours of gameplay.
	
	\section{Deadlines}
	\begin{itemize}
	\item Basic Card System + Basic World Generation (7th June)
	\item Enemy AI and Card battles, World Population, Player Movement (14th June)
	\item Card Upgrades, Merging (20th June)
	\item Lore (30th June) [Hopefully]
	\end{itemize}
\end{document}
